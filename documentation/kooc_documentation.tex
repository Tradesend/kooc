\documentclass{article}

\usepackage[bookmarks]{hyperref}
\usepackage{fancyhdr} % Required for custom headers
\usepackage{lastpage} % Required to determine the last page for the footer
\usepackage{extramarks} % Required for headers and footers
\usepackage[usenames,dvipsnames]{color} % Required for custom colors
\usepackage{graphicx} % Required to insert images
\usepackage{listings} % Required for insertion of code
\usepackage{courier} % Required for the courier font
\usepackage{lipsum} % Used for inserting dummy 'Lorem ipsum' text into the template

% Margins
\topmargin=-0.45in
\evensidemargin=0in
\oddsidemargin=0in
\textwidth=6.5in
\textheight=9.0in
\headsep=0.25in

\linespread{1.1} % Line spacing

% Set up the header and footer
\pagestyle{fancy}
\lhead{} % Top left header
\chead{\hmwkProject: \hmwkTitle} % Top center head
\rhead{\firstxmark} % Top right header
\lfoot{\lastxmark} % Bottom left footer
\cfoot{} % Bottom center footer
\rfoot{Page\ \thepage\ of\ \protect\pageref{LastPage}} % Bottom right footer
\renewcommand\headrulewidth{0.4pt} % Size of the header rule
\renewcommand\footrulewidth{0.4pt} % Size of the footer rule

\setlength\parindent{0pt} % Removes all indentation from paragraphs

%----------------------------------------------------------------------------------------
%	CODE INCLUSION CONFIGURATION
%----------------------------------------------------------------------------------------
\definecolor{MyDarkGreen}{rgb}{0.0,0.4,0.0} % This is the color used for comments
\lstloadlanguages{C} % Load Perl syntax for listings, for a list of other languages supported see: ftp://ftp.tex.ac.uk/tex-archive/macros/latex/contrib/listings/listings.pdf
\lstset{language=C, % Use C in this example
        frame=single, % Single frame around code
        basicstyle=\small\ttfamily, % Use small true type font
        keywordstyle=[1]\color{Blue}\bf, % Perl functions bold and blue
        keywordstyle=[2]\color{Purple}, % Perl function arguments purple
        keywordstyle=[3]\color{Blue}\underbar, % Custom functions underlined and blue
        identifierstyle=, % Nothing special about identifiers                                         
        commentstyle=\usefont{T1}{pcr}{m}{sl}\color{MyDarkGreen}\small, % Comments small dark green courier font
        stringstyle=\color{Purple}, % Strings are purple
        showstringspaces=false, % Don't put marks in string spaces
        tabsize=5, % 5 spaces per tab
        %
        % Put standard Perl functions not included in the default language here
        morekeywords={@new, @delete, @ctor, @dtor},
        %
        % Put Perl function parameters here
        morekeywords=[2]{@namespace, @implementation, @definition, @class},
        %
        % Put user defined functions here
        morekeywords=[3]{@import},
       	%
        morecomment=[l][\color{Blue}]{...}, % Line continuation (...) like blue comment
        numbers=left, % Line numbers on left
        firstnumber=1, % Line numbers start with line 1
        numberstyle=\tiny\color{Blue}, % Line numbers are blue and small
        stepnumber=5 % Line numbers go in steps of 5
}

% Creates a new command to include a perl script, the first parameter is the filename of the script (without .pl), the second parameter is the caption
\newcommand{\KoocCode}[2]{ 
  \begin{itemize}
  \item[]\lstinputlisting[caption=#2,label=#1]{#1.kc}
  \end{itemize}
}
\newcommand{\KoocHeader}[2]{ 
  \begin{itemize}
  \item[]\lstinputlisting[caption=#2,label=#1]{#1.kh}
  \end{itemize}
}

\newcommand{\CCode}[2]{
  \begin{itemize}
  \item[]\lstinputlisting[caption=#2,label=#1]{#1.c}
  \end{itemize}
}
\newcommand{\CHeader}[2]{
  \begin{itemize}
  \item[]\lstinputlisting[caption=#2,label=#1]{#1.h}
  \end{itemize}
}

%----------------------------------------------------------------------------------------
%	DOCUMENT STRUCTURE COMMANDS
%	Skip this unless you know what you're doing
%----------------------------------------------------------------------------------------

%HyperRef Settings
\hypersetup{pdftex,colorlinks=true, allcolors=black}

% Header and footer for when a page split occurs within a problem environment
\newcommand{\enterFeature}[1] {
  \nobreak\extramarks{#1}{#1}\nobreak
  \nobreak\extramarks{#1}{#1}\nobreak
}

% Header and footer for when a page split occurs between problem environments
\newcommand{\exitFeature}[1]{
  \nobreak\extramarks{#1}{#1}\nobreak
  \nobreak\extramarks{#1}{}\nobreak
}

\setcounter{secnumdepth}{0} % Removes default section numbers
\newcounter{FeatureCounter} % Creates a counter to keep track of the number of problems

\newcommand{\FeatureName}{}
\newenvironment{Feature}[1][Problem \arabic{FeatureCounter}]{ % Makes a new environment called homeworkProblem which takes 1 argument (custom name) but the default is "Problem #"
  \stepcounter{FeatureCounter} % Increase counter for number of problems
  \renewcommand{\FeatureName}{#1} % Assign \homeworkProblemName the name of the problem

  \section{\FeatureName} % Make a section in the document with the custom problem count
  \enterFeature{\FeatureName} % Header and footer within the environment
}{
  \exitFeature{\FeatureName} % Header and footer after the environment
}

\newcommand{\FeatureSectionName}{}
\newenvironment{FeatureSection}[1][Test]{ % New environment for sections within homework problems, takes 1 argument - the name of the section
  \renewcommand{\FeatureSectionName}{#1} % Assign \homeworkSectionName to the name of the section from the environment argument
  \subsection{\FeatureSectionName} % Make a subsection with the custom name of the subsection
  \enterFeature{\FeatureName\ [\FeatureSectionName]} % Header and footer within the environment
}{
  \enterFeature{\FeatureName} % Header and footer after the environment
}

%----------------------------------------------------------------------------------------
%	NAME AND CLASS SECTION
%----------------------------------------------------------------------------------------

\newcommand{\hmwkTitle}{Conception} % Assignment title
\newcommand{\hmwkProject}{Kooc} % Course/class
\newcommand{\hmwkAuthorName}{Colliot Vincent, Diego Moran, Frederic Lavrut} % Your name

%----------------------------------------------------------------------------------------
%	TITLE PAGE
%----------------------------------------------------------------------------------------

\title{
  \vspace{2in}
  \textmd{\textbf{\hmwkProject:\ \hmwkTitle}}\\
  \vspace{3in}
}

\author{\textbf{\hmwkAuthorName}}
\date{} % Insert date here if you want it to appear below your name

%----------------------------------------------------------------------------------------

\begin{document}

\maketitle

%----------------------------------------------------------------------------------------
%	TABLE OF CONTENTS
%----------------------------------------------------------------------------------------

%\setcounter{tocdepth}{1} % Uncomment this line if you don't want subsections listed in the ToC

\newpage
\tableofcontents
\newpage

%----------------------------------------------------------------------------------------
%       Mangling
%----------------------------------------------------------------------------------------

\begin{Feature}[Mangling]
  \begin{FeatureSection}[What]
    Mangling/Name Mangling is a technique used to avoid many problem on unique names resolution. \newline
    Mangling will modify the name following some pattern in order to avoid name collisions.
  \end{FeatureSection}
  \begin{FeatureSection}[Why]
      We needed to solve unique name resolution to be able to create a C oriented object. \newline
      When you create a Class in cpp you'r going to set variable and methode in it, if we want to simulate this we need to tell in c that a function is only callable by the struct. The solution for this is to change the function name to className_functionName. \newline
      The function can in fact still be called by the user. This is why we obfuscate the name.
  \end{FeatureSection}
  \begin{FeatureSection}[Code Equivalence]
    \KoocHeader{mangling}{ various examples in Kooc }
    \CHeader{mangling}{ various examples in C }
  \end{FeatureSection}
  \begin{FeatureSection}[Implementation]
    \_\_\_[\{nb\}NAMESPACE\_\_]*[\{nb\}CLASS\_\_]?\{nb\}SYMTYPE\_\_\{nb\}SYMNAME[\_\_[\{nb\}TYPARG\_\_]*\{nb\}RETTYPE]?\_\_ \newline
    \{nb\} = len(word) \newline
    When we mangle we add ___ as separator at the begining and __ between informations, before all informations we tell the lenght of the information.
    We mangle with all the namespaces, the class if present. Symtype makes the difference between a function or a variable. Symname is the name given to this symbol. Then, if it's a function, we also mangle with every argument's type, and the return type.
    This is implemented by a Class who have for purpose to change the variable name like the previous declaration.
  \end{FeatureSection}
\end{Feature}
\clearpage

%----------------------------------------------------------------------------------------
%	Import
%----------------------------------------------------------------------------------------

\begin{Feature}[Import]
  \begin{FeatureSection}[What]
    To support the use of import in our kooc we use an @Import Statement for safe inclusion. \newline
  \end{FeatureSection}
  \begin{FeatureSection}[Why]
    We needed to be able to include header in our languague so we used @Import as statement to simulate the #include.
  \end{FeatureSection}
  \begin{FeatureSection}[Code Equivalence]
    \KoocCode{import}{ Import demo in Kooc code }
    \CCode{import}{ Import translation in C code }
  \end{FeatureSection}
  \begin{FeatureSection}[Implementation]
    To implement this we created a custom node (Imp). \newline
    We pars the statement to change it to a correct C #include. \newline
    This node is not a part of our work tree and only implement a new meta to C methods. \newline
  \end{FeatureSection}
\end{Feature}
\clearpage

%----------------------------------------------------------------------------------------
%	Namespace
%----------------------------------------------------------------------------------------

\begin{Feature}[Namespace]
  \begin{FeatureSection}[What]
    The purpose of a namespace is to define sub context for declaration of variables and functions. \newline
  \end{FeatureSection}
  \begin{FeatureSection}[Why]
    To be able to use our languague as an object oriented languague we needed to add a Namespace declaration. \newline
  \end{FeatureSection}
  \begin{FeatureSection}[Code Equivalence]
    \KoocHeader{namespace}{ Namespace demo in Kooc header }
    \CHeader{namespace}{ Namespace translation in C header }
    \KoocCode{namespace}{ Namespace demo in Kooc code }
    \CCode{namespace}{ Namespace translation in C code }
  \end{FeatureSection}
  \begin{FeatureSection}[Implementation]
    The definition of a namespace (called 'module' in the subject) is @namespace(NAME)\{\}, and its implementation is @definition(NAME)\{\} \newline
    We implemented this by detecting Namespace declaration (@definition) and added a node to do all mangling and operation of transformation for a Namespace declaration. \newline
  \end{FeatureSection}
\end{Feature}
\clearpage

%----------------------------------------------------------------------------------------
%	Class
%----------------------------------------------------------------------------------------

\begin{Feature}[Class]
  \begin{FeatureSection}[What]
    A class is an object containing variables and methods. Inheritance and polymorphism is a possibility of classes. \newline
  \end{FeatureSection}
  \begin{FeatureSection}[Why]
    To be able to use our languague as an object oriented languague we needed to add a Class declaration. \newline
  \end{FeatureSection}
  \begin{FeatureSection}[Code Equivalence]
    \KoocHeader{class}{ Class demo in Kooc header }
    \CHeader{class}{ Class translation in C header }
    \KoocCode{class}{ Class demo in Kooc code }
    \CCode{class}{ Class translation in C code }
  \end{FeatureSection}
  \begin{FeatureSection}[Implementation]
    The definition of a class is @class(NAME)\{\}, and its implementation is @implementation(NAME)\{\} \newline
    Methodes and variable
    Like the previous implementation of Namespace we created a node to do the specific mangling and transformation of the code. \newline
  \end{FeatureSection}
\end{Feature}
\clearpage

%----------------------------------------------------------------------------------------

\end{document}
